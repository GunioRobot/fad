\documentclass{article}

\usepackage{hcar}
\usepackage{paralist}
\urlstyle{tt}

\begin{document}

\begin{hcarentry}{fad: Forward Automatic Differentiation}
\report{Bj\"orn Buckwalter}%05/09
\status{active}
\participants{Barak A.\ Pearlmutter, Jeffrey Mark Siskind}% optional
\makeheader

%(WHAT IS IT?)

Fad is an attempt to make as comprehensive and usable a forward
automatic differentiation (AD) library as is possible in Haskell.  Fad
\begin{inparaenum}[(a)]
\item attempts to be correct, by making it difficult to accidentally
  get a numerically incorrect derivative;
\item provides not only first-derivatives, but also a lazy tower of
  higher-order derivatives;
\item allows nested use of derivative operators while using the type
  system to reject incorrect nesting (perturbation confusion);
\item attempts to be complete, in the sense of allowing calculation of
  derivatives of functions defined using a large variety of Haskell
  constructs; and
\item tries to be efficient, in the sense of both the defining
  properties of forward automatic differentiation and in keeping the
  constant factor overhead as low as possible.
\end{inparaenum}

%(WHAT IS ITS STATUS? / WHAT HAS HAPPENED SINCE LAST TIME?)

Version 1.0 of fad was uploaded to Hackage on April 3. Recent changes
can be found via \texttt{git clone} \url{git://github.com/bjornbm/fad.git}

%(CAN OTHERS GET IT?)

%(WHAT ARE THE IMMEDIATE PLANS?)

\FurtherReading
  \url{http://github.com/bjornbm/fad}
\end{hcarentry}

\end{document}
